\section{IBM Watson}
\index{IBM Watson}
\index{Watson}

\subsection{Old: IBM Watson}

IBM Watson is a super computer built on
cognitive technology that processes information like the way human
brain does by understanding the data in a natural language as well
as analyzing structured and unstructured data \cite{www-ibmwatson-wiki}.
It was initially
developed as a question and answer tool more specifically to
answer questions on the quiz show *Jeopardy* but now it has been
seen as helping doctors and nurses in the treatment of cancer. It
was developed by IBM's DeepQA research team led by David
Ferrucci. With Watson you can create bots that can engage
in conversation with you \cite{www-ibmwatson}. You can
even provide personalized recommendations to Watson by
understanding a user's personality, tone and emotion. Watson uses
the Apache Hadoop framework in order to process the large volume
of data needed to generate an answer by creating in-memory
datasets used at run-time. Watson's DeepQA UIMA (Unstructured
Information Management Architecture) annotators were deployed as
mappers in the Hadoop Map-Reduce framework. Watson is written in
multiple programming languages like Java, C++, Prolog and it runs
on the SUSE Linux Enterprise Server. Today Watson is available as
a set of open source
APIs and Software As a Service product as well\cite{www-ibmwatson}.


\subsection{New: IBM Watson}


The XD Metrics Service (XMS) \cite{hid-sample-vonLaszewski15tas} is a
renewed project of the Technology Audit Service (TAS), which aims at
improving the operational efficiency and management of NSF's XD
network of computational resources. XMS builds on and expands the
successes of the TAS project, such as the development of the XDMoD
tool \index{XDMoD}. This tool provides stakeholders of XD and its
largest project, XSEDE, with ready access to data about utilization,
performance, and quality of service for XD resources and XSEDE-related
services. While the initial project focus was the XD community, the
ongoing effort realized that such a resource management tool would
also be of great utility to high performance computing centers in
general, as well as to other data centers managing complex IT
infrastructure. To pursue this opportunity, Open XDMoD was being
developed, which is an open source version of the tool. Open XDMoD is
already in use by numerous academic and industrial HPC centers. The
XMS project expands XDMoD beyond its original goals, so as to increase
its utility to XD and move it into the realm of a comprehensive
resource management tool for cyberinfrastructure. One example is the
incorporation of job-level performance data through
\textit{TACC\_Stats} into XDMoD\index{TACC\_Stat}. This functionality
provides XDMoD with the ability to identify poorly performing
applications, improve throughput, characterize the system's workload,
and provide metrics critical for the specification of future resource
acquisitions. Given the scale of today's HPC systems, even modest
increases in throughput can have a substantial impact on science and
engineering research. For example, with respect to the XD network,
every 1\% increase in system performance translates into an additional
15 million CPU hours of computer time that can be allocated for
research.


