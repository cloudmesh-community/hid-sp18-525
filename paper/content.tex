% THIS IS SIGPROC-SP.TEX - VERSION 3.1
% WORKS WITH V3.2SP OF ACM_PROC_ARTICLE-SP.CLS
% APRIL 2009
%
% It is an example file showing how to use the 'acm_proc_article-sp.cls' V3.2SP
% LaTeX2e document class file for Conference Proceedings submissions.
% ----------------------------------------------------------------------------------------------------------------
% This .tex file (and associated .cls V3.2SP) *DOES NOT* produce:
%       1) The Permission Statement
%       2) The Conference (location) Info information
%       3) The Copyright Line with ACM data
%       4) Page numbering
% ---------------------------------------------------------------------------------------------------------------
% It is an example which *does* use the .bib file (from which the .bbl file
% is produced).
% REMEMBER HOWEVER: After having produced the .bbl file,
% and prior to final submission,
% you need to 'insert'  your .bbl file into your source .tex file so as to provide
% ONE 'self-contained' source file.
%
% Questions regarding SIGS should be sent to
% Adrienne Griscti ---> griscti@acm.org
%
% Questions/suggestions regarding the guidelines, .tex and .cls files, etc. to
% Gerald Murray ---> murray@hq.acm.org
%
% For tracking purposes - this is V3.1SP - APRIL 2009

\documentclass{acm_proc_article-sp}

\begin{document}

\title{IBM Cloud}
%\subtitle{[Telemedicine, Personal Monitors, and Privacy Issues]}

\numberofauthors{1} 
\author{
\alignauthor
Bruce Walker\titlenote{MS Data Science Student}\\
       \affaddr{Indiana University}\\
       %\affaddr{Phoenix, AZ}\\
       \email{brucwalk@iu.edu}\\
} 
\date{20 March,2018}


\maketitle
\begin{abstract}

In 2017, IBM fully committed to cloud computing. IBM BlueMix
is now IBM Cloud. The changes go far beyond the name. The new 
platform gives IBM a new, singular way to engage customers. Now, services are available on public or 
private clouds, with added capabilities, including database, 
artificial intelligence, and blockchain.

In IBM Cloud, many popular services and applications are 
available, with public or private access. No cloud presence 
is possible without a strong network. IBM has included an 
industry leading level of network capabilities. To ensure 
security, access, and redundancy, IBM operates 60 data centers.

The IBM Cloud connects data science and other tools, such as 
VMware, SAP, Spark, Jupyter, R, and many others. Both open source and proprietary 
applications and services are part of IBM Cloud.
As an industry leader in blockchain, the 
technology is featured in IBM Cloud. Blockchain is becoming 
the most known product IBM offers, and it is a major component
of IBM Cloud. 
\end{abstract}

% A category with the (minimum) three required fields
%\category{H.4}{Information Systems Applications}{Miscellaneous}
%A category including the fourth, optional field follows...

%\terms{Theory}

\keywords{hid-sp18-525, 525, IBM, Cloud} % NOT required for Proceedings

\section{Introduction}

IBM has a long and storied history in the technology world.  
Over a century of products and services are being challenged 
like never before.  In order to survive and thrive, IBM has made 
numerous changes and adaptations have been necessary to achieve
survival.  The latest of those changes has been to an important
product--IBM BlueMix.  To thrive in the current computing 
environment, BlueMix has become IBM Cloud \cite{hid-sp18-525-cloud}. Cloud is a concept that has recently become popular.  With the advent of cloud, many companies have moved operations to the cloud, in order to embrace the new world of data and computing.  IBM has taken steps to enter this world, with the latest steps being the change from BlueMix, to IBM Cloud.  IBM offers the cloud on several different levels and with options to suit any size business.  

This paper focuses on IBM Cloud, and includes  What is the Cloud, from IBM.  Why a move to the cloud is necessary discusses the necessity, when considering that most businesses will have a cloud presence, soon.  Services beyond the cloud are also provided by IBM, such as data analytics.
  

\section{What is the Cloud?}

IBM Cloud includes Infrastructure as a Service (IaaS), Platform 
as a Service (PaaS), and Software as a Service (SaaS).  It also 
includes a comprehensive catalogue of cloud services, such as, 
Watson and Blockchain, that can be easily integrated with the 
PaaS and IaaS, that can help with building modular applications. 

IBM Cloud is built for any sized business from small to very large, 
to easily consume and manage.  There are built-in governance 
capabilities that allow a company to easily manage cloud 
consumption and billing.  IBM takes security very seriously, 
and has instituted very stringent security and compliance based 
rules into the architecture. IBM's Cloud resides in over 60 
data centers, and maintains a global network that connects the data 
centers and additional points of presence.

IaaS, in the IBM Cloud,
includes your choice of virtual servers, for various operating systems
and baremetal servers.  PaaS offers the ability to build applications
in a Cloud Foundry environment or IBM Cloud Container Service, for 
the ability to build Docker Containers for application placement.

SaaS in IBM Cloud is rapidly growing as IBM continues migrate their
software applications from an on premise licensing model, to a cloud-
based consumption model \cite {hid-sp18-525-overview}.


\section{Why Cloud}

As society becomes more connected, the move to the cloud is essential to  organizational success.  IBM is a leader in that transformation.  
The development of services to meet the needs of Internet of Things (IoT), for example, is just one area of IBM specialization. The collection and analysis of many myriad of data streams from sensors in manufacturing equipment, on warehouse employees, and tags on shipping containers, allow the ability to proactively act on the data \cite{hid-sp18-525-winning}.  The cloud enables to easily collect, manage, and maintain all of these devices, seamlessly. 

IoT is one of many emerging technologies that the cloud helps to make easily consumable. Some other emerging technologies enabled by the cloud are Artificial Intelligence, Cognitive, and Blockchain \cite{hid-sp18-525-winning}.  All of these solutions are available in the IBM Cloud, and can be easily deployed for both production and non production environments.  According to IBM, by 2020, 90 percent of of business models may be driven by the cloud with a global cloud market valuation of 250 billion dollars  \cite{hid-sp18-525-winning}.

Beyond emerging technology availability in the cloud, there are several additional benefits that companies can achieve.  The ability for companies to use technology to differentiate and be the disrupter is often constrained by a company's own information technology (IT) organizations.  IT departments, historically, want to control everything from infrastructure to the building of applications, along with the consumption of data.  Cloud offers a different way to approach technology, by offering IT departments the ability to provide governed, self-service technology to the remainder of the business.  This allows the various departments within a company to build and scale business relevant applications under IT's guidance and governance, but without the long time constraints that are typically in place. Governed self-service IT will allow companies to innovate and adopt emerging technologies, at a much faster pace, with the goal of leading or remaining competitive \cite{hid-sp18-525-overview}.


\section{Offerings}

IBM will help drive business models with several cloud offerings.  Some of the IBM Cloud offerings include the IBM Public Cloud, and the IBM Cloud Private \cite{hid-sp18-525-winning}.  

The IBM Public Cloud, formerly called BlueMix is the most widely used IBM cloud platform and is available for all customers to consume, even, in low quantities, at no charge. In larger quantities, IBM Public Cloud is billed on a usage basis, with customer ability to prepay for services on an annual basis for discounts.  

IBM Cloud Private is a private cloud offering that can be deployed to a customer owned data center on customer owned hardware and infrastructure.  This allows customers the choice of maintaining control, especially for those customers that require specific security, compliance, and or governance adherence.  IBM Cloud Private offers customers access to similar services and solutions as IBM Public Cloud, however, it also allows customers the ability to easily deploy other IBM solutions, that would typically reside within customer data centers, such as integration and application development platforms \cite{hid-sp18-525-winning}.  

IBM offers the ability to easily integrate the IBM Cloud Private platform with any public cloud platform, including IBM Public Cloud, Amazon, and Microsoft Azure, among many others \cite{hid-sp18-525-winning}.  This is especially important as most companies will see value and need to adopt solutions from many different cloud providers, as they build out their hybrid cloud infrastructures.  Even IBM realizes that it is unrealistic and naive to think that a company would only consume solutions or services from a single cloud provider.    

\section{Beyond Cloud}

Cloud providers, such as IBM, offer cloud as the platform to consume many different offerings and solutions, however, there is still the data that is generated by these services that is the resource of intelligence of companies.  How companies use this data will be the differentiating factor in many cases, that will dictate customer success or disruption.

IBM also offers solutions for customers who do not want to fully embrace the cloud.  For those customers, they may already have an on premise data center or other equipment.  IBM offers a hybrid cloud solution for those customers.  These customers can get the benefits of both worlds with IBM.  Local applications and equipment working in the Public or Private Cloud with IBM.  

Hybrid cloud lowers the cost of ownership and can be more efficient.  And, because the customer still has some control, because of having local equipment in their own premise, a higher sense of satisfaction is possible, as well \cite{hid-sp18-525-hybrid}.  

The extreme growth of data, which is enabled by cloud computing, has driven the need for data scientists with the ability to perform tasks like predictive and prescriptive analytics, along with machine learning and deep learning models against this data.  

Using machine learning and deep learning models goes beyond what companies have typically analyzed.  It allows them to look at historical data, for example, in a different way, to help make predictions on what will happen in the future, versus just looking at the past.  It also allows a company to perform actions on these predictions and thus, the ability to comprehensively affect process efficiencies.  
 
 \section{Conclusion}
 IBM offers a large number of services, from software and platform as services, to private cloud access.  The offerings are scaleable for any size organization.  IBM has embraced the new world of data by developing enhanced methods of exploring and managing that data.  Data center development has been a vital part of IBM Cloud strategy.  Customers have the opportunity to house hardware in IBM data centers or just utilize the Public or Private Cloud.  Customer flexibility is a key aspect of IBM Cloud.  Cloud based exploration and management is a relatively new frontier, which IBM has endeavored to provide world class service.  With Watson, IBM offers the highest level of data exploration and management, including real time data processing.  Based on significant Cloud offerings, IBM is poised, as an industry leader in the cloud for decades to come, in continuation of a long, rich tradition of business excellence.
 
 
\bibliographystyle{abbrv}
\bibliography{hid-sp18-525}

%\balancecolumns 

\end{document}
